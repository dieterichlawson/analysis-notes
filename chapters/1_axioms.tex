\section{Foundations of the Reals}
  \subsection{The Field Axioms}
    The field axioms are a set of axioms that we accept as the foundation of the reals.\\\\
    $\forall a,b,c \in \reals$:\\

    \textbf{F1 Commutativity}   $a+b = b+a$\\
    \textbf{F2 Associativity}  $a+(b +c) = (a+b)+c$\\
    \textbf{F3 Distributive}  $a(b +c) = ab+ac$\\
    \textbf{F4 Identity}  $\exists\;0,1$ such that $0 + a = a$, $1 \cdot a = a$ \\
    \textbf{F5 Additive Inverse}  $\exists\;-a$ such that $a + (-a) = 0$\\
    \textbf{F5 Multiplicative Inverse}  $\exists\;1/a$ such that $a(1/a) = 1$
  \subsection{The Order Axioms}
    \textbf{O1 Positive Numbers} $\exists$ a set $P \subset \mathbf \reals$ such that for all 
    $a \in \reals$ either $a \in P$, $-a \in P$, or $a = 0$.\\
    \textbf{O2} $a,b \in P$ implies that $a\cdot b$ and $a + b$ are in $P$\\

    Thus we define $a > b$ as $a - b \in P$ and similarly $a < b$ is defined as $b - a \in P$.\\

    $F$ and $O$ axioms hold for the rationals, $\rationals$, but $O$ does not hold for the 
    complex numbers.
  \subsection{The Completeness Axiom}
    Completeness distinguishes the Reals from the Rationals. Intuitively, there are 'holes' in 
    the rationals at irrational numbers like $\sqrt{2}$. To discuss completeness, we need to 
    introduce some definitions.\\

    Consider a set $S$, such that $S \subset \reals$.\\
    $S$ is \emph{bounded above} if $\exists$ $a \in \reals$ such that $x \leq a$ $\forall x \in S$.\\
    $S$ is \emph{bounded below} if $\exists$ $a \in \reals$ such that $x \geq a$ $\forall x \in S$.\\
    $S$ is \emph{bounded} if it is bounded above and below.\\

    \textbf{The Completeness Axiom} If $S \subset \reals$ is nonempty and bounded above then 
    $\exists$ $a \in \reals$ that is a least upper bound or supremum. Specifically, (i) 
    $x \leq a$ $\forall x \in S$ and (ii) $a \leq \beta$ $\forall$ upper bounds, $\beta$, of $S$.\\

    Supremums are unique by (ii) because if $a_1,a_2$ are upper bounds and $a_1 \leq a_2$ and 
    $a_2 \leq a_1$ then $a_1 = a_2$. Thus it makes sense to talk about "the" supremum.\\

    There is also an "infimum" or greatest lower bound that follows from repeating these 
    arguments with $-S$.\\

    \textbf{Note:} It is important to note that the maximum and supremum of a set are not 
    necessarily the same.\\

    The maximum is defined as $a \in S$ such that $x \leq a$ for all $x \in S$\\
    The supremum is defined as $a$ such that $x \leq a$ for all $x \in S$ and (ii)\\

    The supremum does not have to be in $S$. In fact the max of $S$ exists if and only if the 
    supremum of $S$ is a member of $S$, in which case the max of $S$ is equal to the supremum 
    of $S$. Conversely, if $\sup S \notin S$ then the max of $S$ does not exist.\\

    For example take $S = (0,1)$. Then $\sup S = 1$ but $\sup S \notin S$, so $\max S$ does not
    exist. The sequence $1 - 1/n$ for $n \in \naturals$ comes arbitrarily close to the max of 
    $S$ but never reaches it.\\

    However, any \emph{finite}, nonempty set has a maximum.
  \subsection{Consequences of Completeness}
    \subsubsection{Completeness does not hold for $\rationals$}
      Consider the set $\{x \mid x^2 < 2\}$. The number $\sqrt{2}$ is not a member of 
      $\rationals$ so the supremum of this set cannot be a member of the set. Thus $\rationals$ 
      is not complete, i.e. there are 'holes' at the irrational numbers.
    \subsubsection{The Archimidean Property of the Reals}
      \todo[inline]{Prove the Archimidean property of the Reals}
    \subsubsection{$\mathbf{Q}$ and $\mathbf{I}$ are dense in $\mathbf{R}$}
      \todo[inline]{Prove that $\mathbf{Q}$ and $\mathbf{I}$ are dense in $\mathbf{R}$}
    \subsubsection{$\mathbf{Q}$ is countable, $\mathbf{R}$ and $\mathbf{I}$ are uncountable}
      \todo[inline]{Prove that $\mathbf{Q}$ is countable, $\mathbf{R}$ and $\mathbf{I}$ are
      uncountable}
