\section{Real Sequences}
  \subsection{Definitions}
    A sequence is a mapping from $\naturals \to \reals$ and is generally written as $\{a_n\}$.\\

    A sequence is \emph{increasing} if $a_{n+1} \geq a_n$ for all $n \in \naturals$\\
    A sequence is \emph{decreasing} if $a_{n+1} \leq a_n$ for all $n \in \naturals$\\
    A sequences is \emph{strictly} increasing or decreasing if equality never holds.\\
    A sequence is \emph{monotone} if it is increasing or decreasing.\\

    \todo[inline]{Define subsequences}
  \subsection{Convergence}
    A sequence, $\{a_n\}$ is convergent if $\exists\;\ell \in \reals$ such that $\forall \epsilon 
    > 0$ $\exists N$ such that $|a_n - \ell| < \epsilon$ $\forall n \geq N$. $\ell$ is called the
    limit of $\{a_n\}$.\\
    
    In words, this means that a sequence is convergent if for any positive number epsilon we can
    pick a point in the sequence sufficiently far out such that all elements of the sequence 
    after that point are within $\epsilon$ of $\ell$. $\epsilon$ could be any positive number, 
    but the idea is that as $\epsilon$ becomes arbitrarily small, we can find points of the 
    sequence that are arbitrarily close to $\ell$.\\

    In general, proofs of convergence will follow a challenge-response format where given an 
    $\epsilon$ you construct an $N$ such that the criterion holds.\\

    \todo[inline]{Define $\limsup$ and $\liminf$}
  \subsubsection{Any convergent sequence is bounded}
    \thm Any convergent sequence is bounded.\\

    \pf Let $\{a_n\}$ be a convergent sequence with limit $L$. Then there exists an $N$ such that
    for $n \geq N$, $|a_n- L| < \epsilon$, which implies $a_n < L + \epsilon$ for $n \geq N$. 
    Because $N$ is finite, we then know that $a_n \leq \max(a_1,a_2,\ldots,a_{N-1},L+\epsilon)$ 
    for some $N$ and $\epsilon$. Thus $\{a_n\}$ is bounded. $\blacksquare$
  \subsubsection{A bounded, monotone sequence converges}
    \thm A bounded, monotone sequence converges\\

    \pf Assume a sequence, $\{a_n\}$, is increasing (WLOG) and bounded, and let $\ell = \sup S$
    where $S = \{a_n \mid n \in \naturals\}$ (i.e. the set of elements of the sequence). We claim
    that $\limninf a_n = \ell$, or equivalently that $\{a_n\}$ converges to the limit $\ell$.\\

    Let $\epsilon > 0$. Then it must be that $\ell - \epsilon$ is \emph{not} and upper bound 
    because $\ell$ is the supremum of $S$. Thus there exists $a_N \in S$ such that $a_N > \ell
    - \epsilon$. This implies that for all $n > N$, $a_n \geq a_N > \ell - \epsilon$.\\

    On the other hand, because all $a_n$ are members of $S$ and $\ell$ is the supremum, $a_n < 
    \ell + \epsilon$ for all $n \geq N$. Thus for $n \geq N$, $a_n > \ell - \epsilon$ and $a_n
    < \ell + \epsilon$, which implies that $|a_n - \ell| < \epsilon$. Therefore by the definition 
    of convergence, the $\{a_n\}$ converges. $\blacksquare$
  \subsubsection{The Squeeze Theorem}
    \todo[inline]{Prove the Squeeze Theorem}

  \subsection{The Bolzano--Weierstrass Theorem}
    The Bolzano-Weierstrass Theorem is of great importance to analysis and states that any 
    bounded sequence has a convergent subsequence.\\
    
    This is not immediately obvious because the sequence $a_n = (-1)^n$ is bounded but does not 
    converge. However, if we take the subsequences of $n_i = 2n$ or $n_i = 2n-1$ for $n \in 
    \naturals$ then we have only the odd or even terms of $\{a_n\}$. Those subsequences consist 
    of only $1$ and $-1$, respectively, and are thus convergent. We will now make this intuition
    more formal.\\

    \thm Any bounded sequence has a convergent subsequence.\\

    \pf Let $\{a_n\}$ be a bounded subsequence. Then there exists a $l,u \in \reals$ such that 
    $l \leq a_n \leq u$ for all $n \in \naturals$. Then we know that $a_n \in [l,u]$ for all 
    positive $n$. Now consider the bisection of this interval into two, giving the intervals:\\
    \begin{align*}
      \left[l, \dfrac{l+u}{2}\right] , \left[\dfrac{l+u}{2},u\right]
    \end{align*}
    Because there are infinitely many terms in $\{a_n\}$, one or both of these intervals must 
    contain infinitely many terms of $\{a_n\}$. Pick one such interval and label it $I_1$, with 
    its endpoints labeled $l_1$ and $u_1$.\\

    Now repeat this process for $I_1$, bisecting it into two closed intervals, picking one 
    subinterval which contains infinitely many members of $\{a_n\}$, and labelling its endpoints
    $l_2$ and $u_2$. Because there are infintely many elements in $\{a_n\}$ it is possible to 
    pick a sequence of closed intervals, $I_n$ such that $I_1 \supset I_2 \supset I_3 \supset 
    \cdots$ where the width of $I_n$ is $\frac{u-l}{2^n}$. Additionally, each of these intervals
    contains infinitely many elements of $\{a_n\}$.\\

    Now choose a positive integer $n_1$ such that $a_{n_1} \in I_1$. Because $I_2$ contains 
    infinitely many elements of $\{a_n\}$, there exists a positive integer $n_2$ such that $n_2 
    > n_1$ and $a_{n_2} \in I_2$. Continue picking elements of $\{a_n\}$ in this way to construct
    a subsequence, $\{a_{n_i}\}$, such that $a_{n_i} \in I_i$ for all $n_i$. We will show that 
    $\{a_{n_i}\}$ converges.\\
    
    \textbf{Ending 1} We know that there must be one element, $x$ in all $I_n$. Let $\epsilon
    > 0$ and pick an interval, $I_N$ such that the width of $I_N$, $\frac{u - l}{2^N}$ is less
    than epsilon, and pick an element, $a_{n_K}$ such that $a_{n_K} \in I_N$. $x$ must be in this
    interval, and by construction $a_{n_i} \in I_N$ for all $n_i > n_K$. Thus $|a_{n_i} - x| < 
    \epsilon$ for $n_i > n_K$ and $\{a_{n_i}\}$ converges to $x$. $\blacksquare$\\

    \textbf{Ending 2} Consider the sequence of upper bounds on these intervals $u_1, u_2, \ldots$,
    and note that they are bounded and decreasing and therefore converge to some limit, $U$. 
    Similarly, the lower bounds of the intervals, $l_1, l_2, \ldots$ are bounded and increasing
    and therefore converge to some limit $L$. Because the width of interval $n$ is
    $\frac{u-l}{2^n}$, $\lim_{n \to \infty} u_n - l_n = 0$. Finally we know that $u_i \geq 
    a_{n_i} \geq l_i$ for all $i$, so by the squeeze theorem $\{a_{n_i}\}$ converges. 
    $\blacksquare$
  \subsection{Cauchy Sequences}
    A sequence $\{a_n\}$ is called Cauchy if for all $\epsilon > 0$ there exists a positive 
    integer $N$ such that $|a_n - a_m| < \epsilon$ for all $n, m \geq N$. Intuitively, this
    says a sequence is Cauchy if it has a tail where the elements are arbitrarily close 
    together. Note that this is not a statement about consecutive elements in $\{a_n\}$, it is
    a statement about all elements past $N$.
  \subsubsection{A sequence converges if and only if it is Cauchy}
    \thm A sequence converges if and only if it is Cauchy.\\

    \pf ($\implies$) Assume that $\{a_n\}$ is a convergent series with limit $L$. Let 
    $\epsilon > 0$ and choose $N$ such that $|a_n - L| < \epsilon/2$ for all $n \geq N$.\\

    Now choose $m,n \geq N$. By the triangle inequality we know that $|a_n - a_m| \leq
    |a_n - L| + |L - a_m| \leq \epsilon/2 + \epsilon/2 = \epsilon$. Thus for all $n, m\geq N$
    , $|a_n - a_m| < \epsilon$, and $\{a_n\}$ must be Cauchy.\\

    ($\impliedby$) Assume that $\{a_n\}$ is a Cauchy sequence. We will show that $\{a_n\}$ is
    convergent in three steps: (1) Show that any Cauchy sequence is bounded. (2) Use 
    Bolzano-Weierstrass to obtain a convergent subsequence. (3) Show that (2) implies that 
    the whole sequence converges.\\

    Take $\epsilon = 1$. Because $\{a_n\}$ is Cauchy we know that there exists an $N$ such
    that $|a_n - a_m| < 1$ for all $n, m \geq N$. This implies for all $n \geq N$:
    \begin{align*}
      |a_n - a_N| < 1\\
      ||a_n| - |a_N|| < 1\\
      |a_n| - |a_N| < 1\\
      |a_n| <  |a_N| + 1
    \end{align*}
    Because $N$ is finite, there are finitely many elements of $\{a_n\}$ where $n < N$, so we
    know that for all $n$ (not just $n \geq N$), $|a_n| \leq \max(|a_1|,|a_2|,\ldots,
    |a_N|+1)$. Thus $\{a_n\}$ is bounded.\\

    Because $\{a_n\}$ is bounded we know by the Bolzano-Weierstrass theorem that there must 
    exist a convergent subsequence of $\{a_n\}$, $\{a_{n_i}\}$ with limit $L$.\\

    Take $\epsilon > 0$. Then we know that there exists an $N$ such that $|a_{n_i} - L| < 
    \epsilon/2$ for all $n_i \geq N$. Additionally, because $\{a_n\}$ is Cauchy we know that
    there exists an $N'$ such that $|a_n - a_m| < \epsilon/2$ for all $n,m \geq N'$. Then, by
    the triangle inequality we know that $|a_n - L| \leq |a_n - a_{n_i}| + |a_{n_i} - L| < 
    \epsilon/2 + \epsilon/2 = \epsilon$. Thus $\{a_n\}$ must converge, and its limit must be
    $L$. $\blacksquare$
