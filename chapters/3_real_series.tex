\section{Real Series}
  A \emph{series} is the sum of the terms of a sequence, and is often written as 
  $\sum_{k=1}^{\infty} a_k$ where $\{a_n\}$ is the related sequence. The sum of the
  first $n$ terms of a sequence, i.e. $s_n = \sum_{k=1}^n a_k$ is called the $n$th partial sum 
  of a sequence and is generally denoted as $\{s_n\}$.\\
  
  If $\{s_n\}$ has a limit $S$, i.e. if the sequence of partial sums converges, then we say 
  that the series $\sum_{k=1}^{\infty} a_k$ converges and has sum $S$. If $\{s_n\}$ does 
  not converge, i.e. does not have a limit, then we say that the series diverges.
  
  \subsection{Convergence of Series}
    \thm If $\sum_{i=1}^{\infty} a_n$ converges, then $\lim a_n = 0$\\

    \pf \todo[inline]{Prove that if $\sum_{i=1}^{\infty} a_n$ converges, then $\lim a_n = 0$}

    \textbf{Note:} The converse of the above statement is not true. For example 
    $\sum_{i=1}^{\infty} 1/n$ does not converge, but $\lim a_n =0$.\\
 
    \thm If $\sum a_n$, $\sum b_n$ converge to $S$ and $T$ respectively, and $\alpha, \beta \in 
    \reals$ then $\lim_{n \to \infty} \sum_{k=1}^n \alpha a_n + \sum_{k=1}^n \beta b_n = \alpha 
    S + \beta T$.\\

    \pf \todo[inline]{Prove linearity of convergence for series}

    \thm If all terms of a series $\sum a_n$ are greater than 0, then $\sum a_n$ converges
    if and only if the sequence of partial sums $\{s_n\}$ is bounded.\\

    \pf \todo[inline]{Prove that if all terms of a series $\sum a_n$ are greater than 0, 
    then $\sum a_n$ converges if and only if the sequence of partial sums $\{s_n\}$ is bounded}
 
  \subsection{Absolute Convergence}
    A series, $\sum_{n=1}^{\infty} a_n$ converges absolutely if $\sum_{n=1}^{\infty} |a_n|$ 
    converges. If a series converges but does not converge absolutely then it is called 
    conditionally convergent. This can happen when terms in the series cancel in the normal case
    but when wrapped in absolute value do not cancel.\\

    You can think of absolute value of a number as a combination of the positive and negative 
    parts. More specifically, $|a_n| = (a_n)^+ + (a_n)^-$. Then you can rewrite 
    $\sum_{n=1}^{\infty} a_n$ as $\sum_{n=1}^{\infty} ((a_n)^+ - (a_n)^-)$. Thus the series
    is absolutely convergent if and only of both $\sum_{n=1}^{\infty} (a_n)^+$ and 
    $\sum_{n=1}^{\infty} (a_n)^-$ converge.\\

    \thm Absolute convergence implies converges.\\

    \pf \todo[inline]{Prove that absolute convergence implies convergence}

  \subsection{Rearrangements}
    A rearrangement of a sequence $\{a_n\}$ is a 1-1 onto mapping $f:\naturals \to \naturals$.
    Intuitively, it's exactly like what it sounds like -- it just changes the ordering of a 
    sequence.\\

    If $\sum_{n=1}^{\infty} a_n$ is conditionally convergent, it is possible to reorder the terms
    and get different limits. Put another way, not all rearrangements converge or converge to 
    the same limit.\\
    \todo[inline]{Find example of series that has rearrangements with different limits}
    \thm If $\sum a_n$ is absolutely convergent then any rearrangement is also absolutely 
    convergent and has the same sum as $\sum a_n$.\\

    \pf \todo[inline]{Prove that if If $\sum a_n$ is absolutely convergent then any rearrangement
    is absolutely convergent, and has the same sum as $\sum a_n$} 
  \subsection{Convergence Tests}
    \subsubsection{Ratio Test}
      \todo[inline]{Write up ratio test}
    \subsubsection{Root Test}
      \todo[inline]{Write up root test}
  \subsection{Power Series}
    A power series is an infinite series of the form
    \begin{align*}
      P(x) = \sum_{n=0}^{\infty} a_nx^n = a_0 + a_1x + a_2x^2 + \cdots + a_nx^n + \cdots
    \end{align*}
    This power series is centered at 0, and is the special case of a power series with an
    arbitrary center, $c$:
    \begin{align*}
      P(x,c) = \sum_{n=0}^{\infty} a_n(x-c)^n = a_0 + a_1(x-c) + a_2(x-c)^2 + \cdots + 
                                                                            a_n(x-c)^n + \cdots
    \end{align*}
    For all power series one of three possibilities holds
    \begin{enumerate}
      \item[(i)] 
        $P(x)$ converges only for $x = 0$. This occurs if $a_n$ grows extremely rapidly. For
        example\\
        \begin{align*}
          \sum_{n=0}^{\infty} n!x^n = 1 + x + 2!x^2 + 3!x^3 + \cdots + n!x^n + \cdots
        \end{align*}
        By the ratio test we have:
        \begin{align*}
          \lim_{n \to \infty} \left|\dfrac{a_{n+1}}{a_n}\right| 
          = \lim_{n \to \infty} \dfrac{(n+1)!x^{n+1}}{n!x^n} 
          = \lim_{n \to \infty} (n+1)x  = \infty
        \end{align*}
        Thus for all non-zero $x$, this power series diverges.
      \item[(ii)] 
        $P(x)$ converges for all $x$. This occurs if the $a_n$ go to 0 quickly, for example
        \begin{align*}
          e^x = \sum_{n=0}^{\infty} \dfrac{x^n}{n!} = 1 + x + \dfrac{x^2}{2!} + \cdots + 
                                                                       \dfrac{x^n}{n!} + \cdots
        \end{align*}
        To see that this power series is convergent we will use the ratio test
        \begin{align*}
          \lim_{n \to \infty} \left|\dfrac{a_{n+1}}{a_n}\right| 
          = \lim_{n \to \infty} \left|\dfrac{x^{n+1}}{(n+1)!}\dfrac{n!}{x^n}\right|
          = \lim_{n \to \infty} \dfrac{x}{n+1} = 0 
        \end{align*}
        Thus the power series for $e^x$ converges for all finite $x$.
      \item[(iii)] 
        There exists $\rho > 0$ such that $P(x)$ converges absolutely for $|x| < \rho$,
        diverges for $|x| > \rho$. When $|x| = \rho$ there is no statement. In this case 
        $\rho$ is known as "the radius of convergence". For example
        \begin{align*}
          \sum_{n=0}^{\infty} 2^nx^n = 1 + 2x + 4x^2 + \cdots + 2^nx^n + \cdots
        \end{align*}
        Using the ratio test we have that:
        \begin{align*}
          \rho = \lim_{n \to \infty} \left|\dfrac{a_{n+1}}{a_n}\right| 
          = \lim_{n \to \infty} \left|\dfrac{2^{n+1} x^{n+1}}{2^nx^n}\right|
          = \lim_{n \to \infty} 2|x|
        \end{align*}
         The series converges when $\rho < 1$, which occurs when $|x| < 1/2$. Technically,
         we should check what happens when $|x| = 1/2$, but spoiler alert: it diverges...
    \end{enumerate}
    
    We can actually prove that one of these three possibilities always holds.\\

    \pf \todo[inline]{Prove 3 possibilities for power sequences}
