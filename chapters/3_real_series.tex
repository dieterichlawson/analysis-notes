\section{Real Series}
  A \emph{series} is the sum of the terms of a sequence, and is often written as 
  $\sum_{k=1}^{\infty} a_k$ where $\{a_n\}$ is the related sequence. The sum of the
  first $n$ terms of a sequence, i.e. $s_n = \sum_{k=1}^n a_k$ is called the $n$th partial sum 
  of a sequence and is generally denoted as $\{s_n\}$.\\
  
  If $\{s_n\}$ has a limit $S$, i.e. if the sequence of partial sums converges, then we say 
  that the series $\sum_{k=1}^{\infty} a_k$ converges and has sum $S$. If $\{s_n\}$ does 
  not converge, i.e. does not have a limit, then we say that the series diverges.
  
  \subsection{Results on Series}
    \thm If $\sum_{i=1}^{\infty} a_n$ converges, then $\lim a_n = 0$\\

    \pf \todo[inline]{Prove that if $\sum_{i=1}^{\infty} a_n$ converges, then $\lim a_n = 0$}
    $\blacksquare$\\

    \textbf{Note:} The converse of the above statement is not true. For example 
    $\sum_{i=1}^{\infty} 1/n$ does not converge, but $\lim a_n =0$.\\

    \thm If $\sum a_n$, $\sum b_n$ converge to $S$ and $T$ respectively, and $\alpha, \beta \in 
    \reals$ then $\lim_{n \to \infty} \sum_{k=1}^n \alpha a_n + \sum_{k=1}^n \beta b_n = \alpha 
    S + \beta T$.\\

    \pf \todo[inline]{Prove linearity of convergence for series}
    $\blacksquare$
  \subsection{Absolute Convergence}
    A series, $\sum_{n=1}^{\infty} a_n$ converges absolutely if $\sum_{n=1}^{\infty} |a_n|$ 
    converges. If a series converges but does not converge absolutely then it is called 
    conditionally convergent. This can happen when terms in the series cancel in the normal case
    but when wrapped in absolute value do not cancel.\\

    You can think of absolute value of a number as a combination of the positive and negative 
    parts. More specifically, $|a_n| = (a_n)^+ + (a_n)^-$. Then you can rewrite 
    $\sum_{n=1}^{\infty} a_n$ as $\sum_{n=1}^{\infty} ((a_n)^+ - (a_n)^-)$. Thus the series
    is absolutely convergent if and only of both $\sum_{n=1}^{\infty} (a_n)^+$ and 
    $\sum_{n=1}^{\infty} (a_n)^-$ converge.\\
  \subsection{Rearrangements}
    A rearrangement of a sequence $\{a_n\}$ is a 1-1 onto mapping $f:\naturals \to \naturals$.
    Intuitively, it's exactly like what it sounds like -- it just changes the ordering of a 
    sequence.\\

    If $\sum_{n=1}^{\infty} a_n$ is conditionally convergent, it is possible to reorder the terms
    and get different limits. Put another way, not all rearrangements converge or converge to 
    the same limit.
  \subsection{Power Series}
