\documentclass[12pt]{article}
\usepackage{fullpage,graphicx,psfrag,amsmath,amsfonts,amssymb,verbatim,titling,todonotes}
\usepackage[small,bf]{caption}
\input defs.tex
\setlength{\parindent}{0pt}
\title{Math 171 Notes}

\begin{document}
\maketitle
\tableofcontents
\listoftodos
\newpage
\section{Foundations of the Reals}
  \subsection{The Field Axioms}
    The field axioms are a set of axioms that we accept as the foundation of the reals.\\\\
    $\forall a,b,c \in \reals$:\\

    \textbf{F1 Commutativity}   $a+b = b+a$\\
    \textbf{F2 Associativity}  $a+(b +c) = (a+b)+c$\\
    \textbf{F3 Distributive}  $a(b +c) = ab+ac$\\
    \textbf{F4 Identity}  $\exists\;0,1$ such that $0 + a = a$, $1 \cdot a = a$ \\
    \textbf{F5 Additive Inverse}  $\exists\;-a$ such that $a + (-a) = 0$\\
    \textbf{F5 Multiplicative Inverse}  $\exists\;1/a$ such that $a(1/a) = 1$
  \subsection{The Order Axioms}
    \textbf{O1 Positive Numbers} $\exists$ a set $P \subset \mathbf \reals$ such that for all $a \in \reals$ either
    $a \in P$, $-a \in P$, or $a = 0$.\\
    \textbf{O2} $a,b \in P$ implies that $a\cdot b$ and $a + b$ are in $P$\\

    Thus we define $a > b$ as $a - b \in P$ and similarly $a < b$ is defined as $b - a \in P$.\\

    $F$ and $O$ axioms hold for the rationals, $\rationals$, but $O$ does not hold for the complex numbers.
  \subsection{The Completeness Axiom}
    Completeness distinguishes the Reals from the Rationals. Intuitively, there are 'holes' in the rationals at
    irrational numbers like $\sqrt{2}$. To discuss completeness, we need to introduce the "supremum".\\

    Consider a set $S$, such that $S \subset \reals$.\\
    $S$ is \emph{bounded above} if $\exists$ $a \in \reals$ such that $x \leq a$ $\forall x \in S$.\\
    $S$ is \emph{bounded below} if $\exists$ $a \in \reals$ such that $x \geq a$ $\forall x \in S$.\\
    $S$ is \emph{bounded} if it is bounded above and below.\\

    \textbf{The Completeness Axiom} If $S \subset \reals$ is nonempty and bounded above then $\exists$ $a \in \reals$
    that is a least upper bound or supremum. Specifically, (i) $x \leq a$ $\forall x \in S$ and (ii) $a \leq \beta$ $\forall$ 
    upper bounds, $\beta$, of $S$.\\

    Supremums are unique by (ii) because if $a_1,a_2$ are upper bounds and $a_1 \leq a_2$ and $a_2 \leq a_1$ then $a_1 = a_2$.
    Thus it makes sense to talk about "the" supremum.\\

    There is also an "infimum" or greatest lower bound that follows from repeating these arguments with $-S$.

    \subsubsection{The maximum and supremum of a set are not necessarily the same}
      It is important to note that the maximum is not necessarily the same thing as the supremum:\\

      The maximum is defined as $a \in S$ such that $x \leq a$ for all $x \in S$\\
      The supremum is defined as $a$ such that $x \leq a$ for all $x \in S$ and (ii)\\

      Note that the supremum does not have to be in $S$. In fact the max of $S$ exists if and only if the supremum of $S$
      is a member of $S$, in which case the max of $S$ is equal to the supremum of $S$. Conversely, if $\sup S \notin S$ then
      the max of $S$ does not exist.\\

      For example take $S = (0,1)$. Then $\sup S = 1$ but $\sup S \notin S$, so $\max S$ does not exist. The sequence
      $1 - 1/n$ for $n \in \naturals$ comes arbitrarily close to the max of $S$ but never reaches it.\\

      However, any \emph{finite}, nonempty set has a maximum.
  \subsection{Consequences of Completeness}
    \subsubsection{Completeness does not hold for $\rationals$}
      Consider the set $\{x \mid x^2 < 2\}$. The number $\sqrt{2}$ is not a member of $\rationals$ so the supremum of this
      set cannot be a member of the set. Thus $\rationals$ is not complete, i.e. there are 'holes' at the irrational numbers.
    \subsubsection{The Archimidean Property of the Reals}
      \todo[inline]{Prove the Archimidean property of the Reals}
    \subsubsection{$\mathbf{Q}$ and $\mathbf{I}$ are dense in $\mathbf{R}$}
      \todo[inline]{Prove that $\mathbf{Q}$ and $\mathbf{I}$ are dense in $\mathbf{R}$}
    \subsubsection{$\mathbf{Q}$ is countable, $\mathbf{R}$ and $\mathbf{I}$ are uncountable}
      \todo[inline]{Prove that $\mathbf{Q}$ is countable, $\mathbf{R}$ and $\mathbf{I}$ are uncountable}
\section{Sequences}
  \subsection{Definitions}
    A sequence is a mapping from $\naturals \to \reals$ and is generally written as $\{a_n\}$.\\

    A sequence is \emph{increasing} if $a_{n+1} \geq a_n$ for all $n \in \naturals$\\
    A sequence is \emph{decreasing} if $a_{n+1} \leq a_n$ for all $n \in \naturals$\\
    A sequences is \emph{strictly} increasing or decreasing if equality never holds.\\
    A sequence is \emph{monotone} if it is increasing or decreasing.\\

    \todo[inline]{Define subsequences}
  \subsection{Convergence}
    A sequence, $\{a_n\}$ is convergent if $\exists\;\ell \in \reals$ such that $\forall \epsilon > 0$ 
    $\exists N$ such that $|a_n - \ell| < \epsilon$ $\forall n \geq N$. $\ell$ is called the limit of $\{a_n\}$.\\
    
    In words, this means that a sequence is convergent if for any positive number epsilon we can pick a 
    point in the sequence sufficiently far out such that all elements of the sequence after that point 
    are within $\epsilon$ of $\ell$. $\epsilon$ could be any positive number, but the idea is that as 
    $\epsilon$ becomes arbitrarily small, we can find points of the sequence that are arbitrarily close 
    to $\ell$.\\

    In general, proofs of convergence will follow a challenge-response format where given an $\epsilon$
    you construct an $N$ such that the criterion holds.\\

    \todo[inline]{Define $\limsup$ and $\liminf$}
  \subsubsection{Any convergent sequence is bounded}
    \thm Any convergent sequence is bounded.\\

    \pf Let $\{a_n\}$ be a convergent sequence with limit $L$. Then there exists an $N$ such that
    for $n \geq N$, $|a_n- L| < \epsilon$, which implies $a_n < L + \epsilon$ for $n \geq N$. Because
    $N$ is finite, we then know that $a_n \leq \max(a_1,a_2,\ldots,a_{N-1},L+\epsilon)$ for some
    $N$ and $\epsilon$. Thus $\{a_n\}$ is bounded. $\blacksquare$
  \subsubsection{A bounded, monotone sequence converges}
    \thm A bounded, monotone sequence converges\\

    \pf Assume a sequence, $\{a_n\}$, is increasing (WLOG) and bounded, and let $\ell = \sup S$ where 
    $S = \{a_n \mid n \in \naturals\}$ (i.e. the set of elements of the sequence). We claim that 
    $\limninf a_n = \ell$, or equivalently that $\{a_n\}$ converges to the limit $\ell$.\\

    Let $\epsilon > 0$. Then it must be that $\ell - \epsilon$ is \emph{not} and upper bound because
    $\ell$ is the supremum of $S$. Thus there exists $a_N \in S$ such that $a_N > \ell - \epsilon$. This
    implies that for all $n > N$, $a_n \geq a_N > \ell - \epsilon$.\\

    On the other hand, because all $a_n$ are members of $S$ and $\ell$ is the supremum, $a_n < \ell + \epsilon$
    for all $n \geq N$. Thus for $n \geq N$, $a_n > \ell - \epsilon$ and $a_n < \ell + \epsilon$, which implies
    that $|a_n - \ell| < \epsilon$. Therefore by the definition of convergence, the $\{a_n\}$ converges. $\blacksquare$
  \subsubsection{The Squeeze Theorem}
    \todo[inline]{Prove the Squeeze Theorem}

  \subsection{The Bolzano--Weierstrass Theorem}
    The Bolzano-Weierstrass Theorem is of great importance to analysis and states that any bounded sequence has a 
    convergent subsequence.\\
    
    This is not immediately obvious because the sequence $a_n = (-1)^n$ is bounded but does not converge. 
    However, if we take the subsequences of $n_i = 2n$ or $n_i = 2n-1$ for $n \in \naturals$ then we have only 
    the odd or even terms of $\{a_n\}$. Those subsequences consist of only $1$ and $-1$, respectively, and are 
    thus convergent. We will now make this intuition more formal.\\

    \thm Any bounded sequence has a convergent subsequence.\\

    \pf Let $\{a_n\}$ be a bounded subsequence. Then there exists a $l,u \in \reals$ such that $l \leq a_n \leq u$ for
    all $n \in \naturals$. Then we know that $a_n \in [l,u]$ for all positive $n$. Now consider the bisection of
    this interval into two, giving the intervals:\\
    \begin{align*}
      \left[l, \dfrac{l+u}{2}\right] , \left[\dfrac{l+u}{2},u\right]
    \end{align*}
    Because there are infinitely many terms in $\{a_n\}$, one or both of these intervals must contain infinitely many
    terms of $\{a_n\}$. Pick one such interval and label it $I_1$, with its endpoints labeled $l_1$ and $u_1$.\\

    Now repeat this process for $I_1$, bisecting it into two closed intervals, picking one subinterval which contains 
    infinitely many members of $\{a_n\}$, and labelling its endpoints $l_2$ and $u_2$. Because there are infintely many
    elements in $\{a_n\}$ it is possible to pick a sequence of closed intervals, $I_n$ such that $I_1 \supset I_2
    \supset I_3 \supset \cdots$ where the width of $I_n$ is $\frac{u-l}{2^n}$. Additionally, each of these intervals
    contains infinitely many elements of $\{a_n\}$.\\

    Now choose a positive integer $n_1$ such that $a_{n_1} \in I_1$. Because $I_2$ contains infinitely many elements 
    of $\{a_n\}$, there exists a positive integer $n_2$ such that $n_2 > n_1$ and $a_{n_2} \in I_2$. Continue picking
    elements of $\{a_n\}$ in this way to construct a subsequence, $\{a_{n_i}\}$, such that $a_{n_i} \in I_i$ for all $n_i$.
    We will show that $\{a_{n_i}\}$ converges.\\
    
    \textbf{Ending 1} We know that there must be one element, $x$ in all $I_i$. Let $\epsilon > 0$ and pick
    an interval, $I_N$ such that the width of $I_N$, $\frac{u - l}{2^N}$ is less than epsilon, and pick an 
    element, $a_{n_K}$ such that $a_{n_K} \in I_N$. $x$ must be in this interval, and by construction 
    $a_{n_i} \in I_N$ for all $n_i > n_K$. Thus $|a_{n_i} - x| < \epsilon$ for $n_i > n_K$ and $\{a_{n_i}\}$
    converges to $x$. $\blacksquare$\\

    \textbf{Ending 2} Consider the sequence of upper bounds on these intervals $u_1, u_2, ... $, and note that
    they are bounded and decreasing and therefore converge to some limit, $U$. Similarly, the lower bounds
    converge to some limit $L$. Because the width of interval $n$ is $\frac{u-l}{2^n}$, $\lim_{n \to \infty} u_n - l_n = 0$.
    Finally we know that $u_i \geq a_{n_i} \geq l_i$ for all $i$, so by the squeeze theorem $\{a_{n_i}\}$ converges. 
    \todo[inline]{Make the alternate ending of BW more rigorous} $\blacksquare$
\end{document}
